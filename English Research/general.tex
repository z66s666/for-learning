\documentclass[12pt,a4paper]{ctexart}
\usepackage{titlesec,graphicx,amsmath,amssymb}
\graphicspath{{figures/}}
\titleformat{\section}{\normalfont\large\bfseries}{\thesection}{1em}{}
\title{
    {\heiti\zihao{2} 大英四研究过程总结}\\
    {\songti\zihao{4} ————互联网抽象文化与网络暴力之间的关系}
}
\author{
    {\songti\zihao{4} 小组成员: 李维奇林, 潘翰霖, 张硕, 张书恺}\\
}
\date{最后更新: 2025.5.5}

\begin{document}
\maketitle
\tableofcontents
\newpage

\section{研究主要介绍}
\subsection{背景}

关键词: 抽象文化, 网络暴力.

(抽象文化经历一个从直播间推广到社交媒体再到整个网络空间的过程, 起源于一些主播为了吸引粉丝、提高热度而作出奇怪的行为, 发表奇怪的言论.
所谓"黑粉"在抽象文化的最初传播中起到了重要的作用, 将他们内部的风波带进了社交媒体如微博, 形成"黑粉"文化, 他们极力地抹黑网络主播等网红, 也就是对主播实施网络暴力, 并将其塑造成"恶棍"、"坏人", 甚至将其视为"反动分子".
为了躲避网络平台的监管, "黑粉"使用抽象话来模糊各种概念, 创造许多简单但内涵丰富的词汇、缩写, 使得这些梗得以在某些社交媒体中传播.
接着, 更加广大的网络用户开始跟风使用这些抽象话, 他们更多只是单纯地玩文字游戏或者是体现自己的"合群", 而不去考虑抽象话最初的意思.
抽象文化从此走进了最广大网络用户的生活, 人们开始习惯于使用抽象话, 在解读这些话的内涵和说话者想要表达的意思中产生欢乐, 在最常见的场景当中, 抽象话、梗的网络暴力成分被逐渐消解、解构, 这些简短的文字具有极高的"超链接"能力, 可解读的自由度极高, 依靠极弱但是有趣的某些关联性深深印在网络用户的脑海里, 见到这些梗人们便会心一笑.
虽然抽象文化中最本质的最目的性的暴力性质在逐渐减弱, 但是这不影响抽象文化的传播, 它消解了网络暴力的特征(说话者是否隐含暴力成分成为了可选择的因素), 抽象文化寄生在日渐普及的网络空间中, 直接影响着网络用户最常用的语言, 并在一定程度上塑造了网络的文化氛围.
抽象文化的影响力在不断增强.)

\subsection{研究的问题}

抽象文化流行的原因与其本身的特点有关吗?

普遍的流行与暴力之间似乎存在着一种负增长的关系, 抽象文化流行的背后是消解暴力特征的过程吗?

各种抽象文化的出现、形式、内容之间有无相似性?

网暴的动机有没有什么相似点?

利用抽象来进行网暴的同时躲避监管是否是最常见的做法?

能否给出明确的定义来区分单纯的玩梗与真正的网暴?(无假设)

怎么反制这些抽象性质的网暴?(无假设)

\subsection{假设}

抽象文化的内容之间存在着某种联系, 容易被人熟悉, 这是抽象文化快速传播的原因.

随着这些抽象词汇和表达的流行, 它们的暴力性会不断减弱, 但抽象文化的影响力不会减弱.

抽象文化的内容都经历了一个相似的产生与变革过程.

大部分网暴都源于未知群体与已知个人/群体之间的利益矛盾, 未知群体利用了这个优势.

利用抽象来进行网暴的同时躲避监管是最常见的做法.

\subsection{研究阶段性目标}

抽象文化内容暴力性与流行性的关系.

抽象文化与网络暴力的定义区分.

网络暴力表达的暴力性解构过程.

非暴力性抽象文化的暴力性重建过程.

(避免网络暴力)

\section{主要过程记录}

\subsection{预实验}

目的: 检验网络抽象文化的信息相通性, 确定大多数网络用户群体对抽象的认知全能.

关键过程: 在多个平台注册一系列新账号, 以注册时的信息为初始参数, 以账号偏向的抽象范围内的图文、视频等信息的类型为过程参数, 账号活跃一段时间, 判断大数据的推送信息是否跨越不同抽象范围.

\subsection{网络爬虫}

目的: 对互联网中尽量大的区域、尽量多的用户的有关抽象文化或网络暴力的行为进行统计, 设置一系列需要研究的参数指标分类统计.

关键过程: 我们选择bilibili网站上关于瓦学弟、牢大内容的图文或者视频下方的评论区, 利用爬虫技术, 获得含有某些人为/主观给定的暴力关键词的评论, 并收集评论者/回复者的一些公开信息, 进行人工分析和机器统计.

\subsection{问卷调查}

目的: 研究不同年龄, 性别, 受教育程度的人群对于抽象文化的了解程度以及他们认知中抽象文化与网络暴力之间的关系.

问题: 略

问卷调查结果(尽量细节):

最终参与调查, 提交问卷的人数: 109.

参与者主要位置分布(从多到少): 广东, 山东, 辽宁, 湖北, 北京, 浙江.

参与者年龄段: 15~20岁占57\%, 21~25岁占33\%.

参与者性别: 男$\frac{4}{5}$, 女$\frac{1}{5}$.

参与者受教育程度: 本科占57\%, 普高占23\%.

参与者开始接触网络的大概年龄: 3~6岁占19\%, 6~12岁占61\%, 12~18岁占16\%.

参与者网龄: 2~4年占18\%, 4~8年占52\%, 8年以上占28\%.

上网频率: 时间越长的选项人数越多, 大致呈二次函数递增趋势.

对抽象文化的理解程度: 约$\frac{2}{3}$的人认为自己经常接触抽象文化, 约$\frac{1}{3}$的人认为自己偶尔接触抽象文化。
绝大部分人了解抽象文化, 一半人对自己的了解程度有自信.
图文、视频网站是抽象文化的主要来源.

从抽象文化小测试中, 参与者对大部分有标准情景梗的题目的选择都契合抽象文化实际, 参与者对绝大部分抽象表达的含义理解都有统一性, 印证了抽象文化在网民中的普及性(幸存者偏差).

参与者遭受网络暴力: 是14\%, 否70\%, 不清楚16\%.

认为抽象文化容易引发暴力的参与者数量是认为不容易引发暴力的参与者数量的1.5倍.
他们认为由于抽象文化的解释灵活性、模糊性导致误解是重要原因, 其他还有心态是否严肃、网络空间鱼龙混杂等.
前三个情景, 参与者依分布的选择结果是轻度暴力, 最后一个情景, 参与者依分布的选择结果是中度暴力.

\end{document}
